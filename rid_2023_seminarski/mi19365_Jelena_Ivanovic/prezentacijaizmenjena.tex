\documentclass[11pt]{beamer}
\usetheme{Madrid}
\usepackage[utf8]{inputenc}

\usepackage[brazil]{babel}

\usepackage{amsmath}
\usepackage{amsfonts}
\usepackage{amssymb}
\usepackage{graphicx}
\DeclareMathOperator {\argmin}{argmin}

\author{Jelena Ivanovic}
\title{ Why shouldn't knoweledge bee free}

\setbeamertemplate{navigation symbols}{} 

\institute[]{Matematički fakultet, Beograd\\Predmet: Računarstvo i društvo\\ Profesor: Dr. Sana Stojanović Đurđević} 
\
\date{Mart 17, 2023} 


\begin{document}

\begin{frame}
\titlepage
\end{frame}

\begin{frame}{Sadržaj}
\tableofcontents 
\end{frame}

\section{Prednosti}
\section{Zaključak}


\begin{frame}{Prednosti}
        \begin{itemize}
            \item Ispravnost sadržaja
            \item Kod besplatne distribucije znanja postoji rizik od netačnih ili većeg broja različitih informacija, dok plaćeni izvori motivišu proveru i osiguranje tačnosti. 
            \item Kada se znanje naplaćuje, ljudi koji ga proizvode i distribuiraju imaju veći interes da osiguraju da je ispravno. Naučni magazini imaju tim ljudi koji proveravaju tačnost informacija, i smanjuje se mogućnost dezinformacija.
        \end{itemize}
       
\end{frame}

\begin{frame}{Prednosti}
        \begin{itemize}
        \item Ekskluzivnost i pravednost u napretku
        \item Naplaćivanje znanja može pomoći u stvaranju ekskluzivnosti i većeg kvaliteta. Naučni časopisi mogu biti motivirani da proizvode i distribuiraju bolje kvalitete znanja kada mogu zaraditi novac od toga. Sa druge strane zbog njihove želje za najbolje kvalitete priznavaće se vrhunski radovi, koji ispunjavaju uslov da se nazivaju naučnim  i napredak u položaju naučnika biće pravedan i po zasluzi.

        \end{itemize}
       
\end{frame}


\begin{frame}{Prednosti}
        \begin{itemize}
           \item Priznanje autorskih prava
           \item Besplatan pristup znanju može dovesti do kršenja autorskih prava. Ako ljudi slobodno dele znanje, to može dovesti do gubitka autorskih prava ali i dobijanja zasluga za tuđe radove.
             \item Za slobodne informacije na internetu nemamo granciju da to nije copy-paste nečijeg rada. Sa druge strane kao i ispravnost dokumenata i informacija takođe naučni magazini vode računa o autorskim pravima i sprečavanju plagijata

        \item Sa otvorenim pristupom neki pojedinac može da okači nečiji rad sa svojim potpisom, zbog nedostatka provere većina ljudi ne bi ni primetilo. Kod naučnih časopisa imamo ceo tim stručnjaka koji proveravaju ispravnost rada, a samim timi to da nisu nečije kopije. Takva greška bi bila fatalna za naučne časopise.


        \end{itemize}
       
\end{frame}

\begin{frame}{Prednosti}
        \begin{itemize}
            \item  Finansiranje i marketing
            \item Znanje zahteva finansiranje za njegovu produkciju i distribuciju.
            \item  Ako je znanje besplatno, ko će finansirati njegovu proizvodnju i održavanje?
            \item Izrada i održavanje web stranica koje sadrže dokumente zahteva finansijski ulog. Na primer, naučnik iz oblasti koja nije informatika bi morao da angažuje i plati stručnjake da mu pomognu u izradi i održavanju svoje web stranice.
            \item Takođe, potrebno je da naučno delo dospelo do velikog broja ljudi, gde naučni časopisi pomažu uz pomoć mailing liste. Mailing lista je selektivna grupa primaoca koji su zainteresovani za određenu temu ili oblast. Korišćenjem mailing liste, izdavači mogu direktno dostaviti informacije i sadržaj relevantnom auditorijumu, čime se olakšava širenje informacija i promocija dela. Naučni časopisi su dobro poznati i omogućavaju mnogim korisnicima da pristupe delima, istovremeno olakšavajući pretragu i dostupnost.
        \end{itemize}
       
\end{frame}

\begin{frame}{Zaključak}
    \begin{itemize}
      \item Vrednost naučnog dela, detaljne i opširne informacije ali pre svega tačne
      \item Pravda u samom poretku naučnika, svako priznato delo nekog naučnika se zaslužno može nazvati naučnim delom i nikako nije površno napisano. Tako da ukupan broj dela nekog naučnika predstavlja realno i istinitno stanje.
    
    \end{itemize}
       
\end{frame}


\end{document}